
\section{\&-Prolog: Independent And-Parallelism}

Parallel execution is a way to speed up programs. In Prolog there are
several possible sources for parallelism. One of them is to execute in
parallel goals (another one will execute clauses in paralell, and
there are other possibilities). This is known as {\bf and-parallelism}.

In \&-Prolog goals which are independent may be executed in parallel.
Goals which are not independent should not be executed in parallel: if
they are, the results are unpredictable. One way to ensure that
parallel goals are independent is to parallelize the program
automatically. This can be done with the Ciao preproccessor: CiaoPP.

\&-Prolog style of parallelism is enabled with the package
\verb+andprolog+, which allows the use of two special constructs for
parallelism. One is the the \verb+&+ and the other the \verb+=>+.

The \verb+&+ is used instead of commas to separate goals which are to
be run in and-parallel. For example, in:
\begin{quote}
\begin{verbatim}
:- use_package(andprolog).

... :- ..., possible(proj) & worth(proj), ...
\end{verbatim}
\end{quote}
%
both goals will be run in parallel. This means that they will be run
by independent Prolog engines, and, if there are available processors,
by different proccesors. Conceptually, two (independent) processes are
``forked'', one for each goal.

The main execution thread, i.e., the clause in which the goals appear,
will be stopped, waiting for completion of the parallel goals. When
these ones finish, their solutions (if they have variables) are
``joined'' together with the main thread, and this one continues
execution. 

The \verb+&+ construction represents, thus, {\em fork-and-join}
parallel expressions. The join is performed without provisions for
solutions given for the same variables by the parallel processes that
may be incompatible. This is the reason for imposing that the goals be
independent. For example:
\begin{quote}
\begin{verbatim}
:- use_package(andprolog).

... :- ..., possible(X) & worth(X), ...
\end{verbatim}
\end{quote}
%
where the goals succeed with \verb+possible(proj)+ and
\verb+worth(any)+ will not work, since the join does not provide means
to deal with the common bindings \verb+X=proj+ and \verb+X=any+.

\subsection{Variable Independence}

Goal independence can be defined from the independence of the terms to
which the arguments of the goals are bound. In turn, this can be
defined in terms of the (program) variables in the goal arguments.

In the above example, the two goals will be independent if they cannot
bind the same variable; since they have variable \verb+X+ in common
(i.e., they {\em share} it), this is only possible if that variable
cannot be bound, i.e., if it is already bound to a term which has no
variables, i.e., if it is already ground (which is what happens in the
first example above). Now consider:
\begin{quote}
\begin{verbatim}
... :- ..., possible(X) & worth(Y), ...
\end{verbatim}
\end{quote}

These two goals will be independent if the terms to which \verb+X+ and
\verb+Y+ are bound do not share any variable. I.e., if the two
variables, in turn, are independent.

To guarantee independence of parallel goals, checks on their argument
variables can be used, as suggested above. These checks are:
\verb+ground/1+ for groundness, \verb+indep/2+ for independence of two
terms (they do not share any variable), and \verb+indep/1+ for the
independence of a list of pairs of terms (the terms in each pair are
independent of each other). 

These checks are available through the use of the \verb+andprolog+
package. Thus, you can write the example above as:
\begin{quote}
\begin{verbatim}
... :- ..., ( indep(X,Y) -> possible(X) & worth(Y)
                          ; possible(X), worth(Y) ), ...
\end{verbatim}
\end{quote}
%
to be sure that parallel execution takes place only if it is safe.

As syntactic sugar for the above conditional expression, the package
also provides the CGE (Conditional Graph Expression) construct, via
the operator \verb+=>+. Thus, you can write the above as:
\begin{quote}
\begin{verbatim}
... :- ..., ( indep(X,Y) => possible(X) & worth(Y) ), ...
\end{verbatim}
\end{quote}
%
where the two goals will be run in parallel if the condition holds,
otherwise they will run sequentially.

The safety condition for parallel execution of any number of goals can
be defined from the conditions on each two goals in turn. That is to
say, a parallel expression with a given number of goals is independent
iff the goals are pairwise independent. E.g., in:
\begin{quote}
\begin{verbatim}
... :- ..., possible(X) & worth(Y) & beneficial(Z) & good(X), ...
\end{verbatim}
\end{quote}
%
the condition will be:
\begin{quote}
\begin{verbatim}
    indep(X,Y), indep(X,Z), indep(X,X), indep(Y,Z)
\end{verbatim}
\end{quote}

Of course, variable independence is commutative, so checks like
\verb+indep(Y,X)+ have not been included. Also, \verb+indep(X,X)+ is
equivalent to \verb+ground(X)+. And this one, in turn, implies the
independence of \verb+X+ from any other term. Thus, the above
condition can be simplified to:
\begin{quote}
\begin{verbatim}
    ground(X), indep(Y,Z)
\end{verbatim}
\end{quote}

The rule of thumb is: any variables shared between any two (or more)
goals have to be ground. All variables not shared between the goals
have to be pairwise independent. 

To check this, you can use a list of the shared variables, and a list
of (all) pairs of non-shared variables. For example, for:
\begin{quote}
\begin{verbatim}
... :- ..., exp(S,P) & cge(S,C) & worth(C) & useful(S,U,F), ...
\end{verbatim}
\end{quote}
%
a minimal condition is:
\begin{quote}
\begin{verbatim}
    ground([S,C]), indep([[P,U],[P,F]])
\end{verbatim}
\end{quote}


\subsection{Constraint Independence}

When running parallel goals in a constraint domain, things get a
little bit more complicated. First, the conditions on the argument
variables are not that simple. Second, for more than two goals, the
condition cannot be defined in terms of pairwise independence.

\comment{Incomplete}
