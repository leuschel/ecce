%
% Brochure for Ciao1.4 publication
%
\documentclass{article}

\usepackage{a4wide}
\usepackage{twocolumn}
\usepackage{html}

%%%%%%%%%%%%%%%%%%%%%%%%%%%%%%%%%%%%%%%%%%%%%%%%%%%%%%%%%%%%%%%%%%%%%%%%%%
%% FORMAT
%%%%%%%%%%%%%%%%%%%%%%%%%%%%%%%%%%%%%%%%%%%%%%%%%%%%%%%%%%%%%%%%%%%%%%%%%%

\usepackage{a4wide}

\topmargin-0.5cm
%% For Apple Laserwriter:
%%\topmargin-3cm
%%\evensidemargin0cm
%%\oddsidemargin0cm
%\textwidth15cm
%\textheight22cm
%%\footskip0.9cm
%% For Duplex in HP Laserwriter:
%\oddsidemargin7mm
%\evensidemargin0mm

\tolerance=10000

\makeatletter
\def\section{\@startsection {section}{1}{\z@}{+1.5ex plus +1.5ex minus 
+1.5ex}{1.5ex}{\Large\bf}}
%\def\subsection{\@startsection{subsection}{2}{\z@}{-1.0ex plus -1ex minus 
%-.2ex}{0.7ex plus .2ex}{\normalsize\bf}}
%\def\subsubsection{\@startsection{subsubsection}{3}{\parindent}{3.25ex plus 
% 1ex minus  .2ex}{-1em minus -.25em}{\normalsize\sc}}
\makeatother
%%% \parskip 6pt
%\parskip 0pt

\def\longrule{\noindent \hrule width \textwidth}

\renewcommand{\baselinestretch}{1.05}
\renewcommand{\topfraction}{0.90}
\renewcommand{\textfraction}{0.10}
\renewcommand{\floatpagefraction}{0.90}

\input{psfig}
\input{/home/clip/Papers/clip_description/logo.tex}
%% Laserprep prologues
%% \newcommand{\psfigpath}{/usr/local/lib//inputs}
%% \newcommand{\lprepold}{\psfigpath/mac.pro}
%% \newcommand{\lprepnew}{\psfigpath/lprep68.pro}

\newcommand{\headline}[1]{\centerline{\underline{\LARGE\bf #1}}}
\newcommand{\head}[1]{\centerline{{\Large\bf #1}}}
\newcommand{\parund}[1]{\paragraph{\underline{#1}}}

\newcommand{\myitems}{
\parsep 0pt
\itemsep 0pt
\topsep 0pt
}

\newcommand{\inbf}[1]{{\bf #1}}
\newcommand{\enfat}[1]{{\em #1}}
\newcommand{\apl}[1]{{\bf #1}}
\newcommand{\concept}[1]{{\em #1}}


\newcommand{\ciao}{\psfig{figure=ciao_s.ps}}

\newcommand{\ciaologo}{
 \vbox{\ciao
  \parbox{4cm}{ {\small \ \\ [-2.4cm]
                   {\bf CLIP} \\ [+0cm]
                   Clip's \\ [-0.2cm]
                   Implementation \\ [-0.2cm]
                   And Other friends \\ [-0.2cm]
                   programming environment}}
  }
}

\begin{document}

\pagestyle{empty}


%% Logos:
\twocolumn[
\ \rule{\textwidth}{.04 in}\\
% Header
\hspace*{0.04\textwidth}
\hbox{
      \vbox{\vspace{12pt}\hbox{\engcliplogo}}
      \hspace*{+0.25\textwidth}
      \vbox{\ciao}
}\\
% Horizontal line
\ \rule{\textwidth}{.04 in}\\ %.01

\begin{center}
\begin{LARGE}
\underline{{\bf Ciao Prolog 1.4 is now available}}\\ % [4mm]
\end{LARGE}
\begin{Large}
{\tt ciao@clip.dia.fi.upm.es}\\
{\tt http://www.clip.dia.fi.upm.es/Software/Ciao}\\ % [2mm]
\end{Large}
\end{center}
]

\noindent{\large\bf Ciao\ldots}

 is a \index{public domain} programming environment which
supports the development and efficient compilation of \enfat{next
  generation} logic programs.

 has been designed from the ground up to be highly extensible (in a
 modular way) and to allow modular program development.

 includes standard \inbf{ISO-Prolog} as a sublanguage, but supports
also logic programming extensions such as 
functional logic programs and object-oriented logic programs. 

 offers support for \enfat{programming in the large} with
      a robust module/object system, module-based automatic
      incremental compilation (with no need for makefiles), an
      assertion language for declaring (\enfat{optional}) program
      properties (including types and modes), automatic static
      inference and static/dynamic checking of such assertions, etc.

 also offers support for \enfat{programming in the small}
      producing small executables (including only those builtins used
      by the program) and support for writing scripts in Prolog.

 includes a compiler which generates several forms of
      architecture-independent and stand-alone executables.  Program
      modules can be compiled into compact bytecode 
      and linked statically, dynamically, or autoloaded.

 supports concurrency (threads), distributed execution (agents),
      and parallel execution.

 programming environment includes a large number of
      \enfat{libraries} providing a wide range of additional
      functionality, from WWW programming to support for several
      control rules.

 programming environment also includes:
      \apl{lpdoc} (distributed separately), a tool for generating
      documentation automatically 
      for Prolog programs adorned with (\inbf{Ciao}) assertions and
      special, machine-readable comments;
      \apl{ciaopp} (the \inbf{Ciao preprocessor}, distributed separately),
      a tool for global program analysis and static debugging and
      optimization via source to source program transformation, including
      automatic parallelization;
      a rich \apl{emacs} interface, and a number of execution
      visualization tools. 

\vfill
\noindent{\large\bf \ldots 1.4\ldots}

 has a much more improved documentation.

 supports automatic compilation of foreign files.

 provides new concurrency primitives and recovers \&Prolog-like 
 multiengine capability. 

 includes a new version of O'Ciao object-oriented library, with
 improved performance. 

 provides support for {\em predicate abstractions}.

 supports reexportation through reexport declarations, precedence of
 importations has changed: last one is now higher. 
 Modules can now implicitly export all predicates.

\vfill
\noindent{\large\bf \ldots and beyond\ldots}

 plans include providing compilation to C source files, support for
constraint logic programming (CLP),

\vfill

%% \inbf{Ciao} is distributed under the \concept{GNU General Public
%% License}, version 2.
%%  as published by the Free Software Foundation,
%% Inc., 675 Mass Ave, Cambridge, MA 02139, USA.

\vfill

\noindent
{\bf Contact info:}
\ \\
%% \begin{itemize}
%% \item Postal address is: \\
%%     \ \\
    \htmladdnormallink{\texttt{http://www.clip.dia.fi.upm.es}}{http://www.clip.dia.fi.upm.es}\\ 
    \htmladdnormallink{\texttt{clip@clip.dia.fi.upm.es}}{mailto:clip@clip.dia.fi.upm.es}\\ 
    Manuel Hermenegildo\\
    Facultad de Inform\'{a}tica -- UPM\\
    E-28660 Boadilla del Monte, Madrid, SPAIN

\end{document}
