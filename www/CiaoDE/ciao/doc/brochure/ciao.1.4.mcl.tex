% Brochure for Ciao1.4 publication
%
\documentclass{article}

\usepackage{a4wide}
\usepackage{twocolumn}
\usepackage{html}

%%%%%%%%%%%%%%%%%%%%%%%%%%%%%%%%%%%%%%%%%%%%%%%%%%%%%%%%%%%%%%%%%%%%%%%%%%
%% FORMAT
%%%%%%%%%%%%%%%%%%%%%%%%%%%%%%%%%%%%%%%%%%%%%%%%%%%%%%%%%%%%%%%%%%%%%%%%%%

\usepackage{a4wide}

\topmargin-0.5cm
%% For Apple Laserwriter:
%%\topmargin-3cm
%%\evensidemargin0cm
%%\oddsidemargin0cm
%\textwidth15cm
%\textheight22cm
%%\footskip0.9cm
%% For Duplex in HP Laserwriter:
%\oddsidemargin7mm
%\evensidemargin0mm

\tolerance=10000

\makeatletter
\def\section{\@startsection {section}{1}{\z@}{+1.5ex plus +1.5ex minus 
+1.5ex}{1.5ex}{\Large\bf}}
%\def\subsection{\@startsection{subsection}{2}{\z@}{-1.0ex plus -1ex minus 
%-.2ex}{0.7ex plus .2ex}{\normalsize\bf}}
%\def\subsubsection{\@startsection{subsubsection}{3}{\parindent}{3.25ex plus 
% 1ex minus  .2ex}{-1em minus -.25em}{\normalsize\sc}}
\makeatother
%%% \parskip 6pt
%\parskip 0pt

\def\longrule{\noindent \hrule width \textwidth}

\renewcommand{\baselinestretch}{1.05}
\renewcommand{\topfraction}{0.90}
\renewcommand{\textfraction}{0.10}
\renewcommand{\floatpagefraction}{0.90}

\input{psfig}
\input{/home/clip/Papers/clip_description/logo.tex}
%% Laserprep prologues
%% \newcommand{\psfigpath}{/usr/local/lib//inputs}
%% \newcommand{\lprepold}{\psfigpath/mac.pro}
%% \newcommand{\lprepnew}{\psfigpath/lprep68.pro}

\newcommand{\headline}[1]{\centerline{\underline{\LARGE\bf #1}}}
\newcommand{\head}[1]{\centerline{{\Large\bf #1}}}
\newcommand{\parund}[1]{\paragraph{\underline{#1}}}

\newcommand{\myitems}{
\parsep 0pt
\itemsep 0pt
\topsep 0pt
}

\newcommand{\inbf}[1]{{\bf #1}}
\newcommand{\enfat}[1]{{\em #1}}
\newcommand{\apl}[1]{{\bf #1}}
\newcommand{\concept}[1]{{\em #1}}


\newcommand{\ciao}{\psfig{figure=ciao_s.ps}}

\newcommand{\ciaologo}{
 \vbox{\ciao
  \parbox{4cm}{ {\small \ \\ [-2.4cm]
                   {\bf CLIP} \\ [+0cm]
                   Clip's \\ [-0.2cm]
                   Implementation \\ [-0.2cm]
                   And Other friends \\ [-0.2cm]
                   programming environment}}
  }
}

\begin{document}

\pagestyle{empty}

%% Logos:
\twocolumn[

% \ \vspace*{-36pt}\\
\ \rule{\textwidth}{.04 in}\\
% Header
\hspace*{0.04\textwidth}
\hbox{
      \vbox{\vspace{12pt}\hbox{\engcliplogo}}
      \hspace*{+0.25\textwidth}
      \vbox{\ciao}
}\\
% Horizontal line
\ \rule{\textwidth}{.04 in}\\ %.01

\begin{center}
\begin{LARGE}
\underline{{\bf Ciao Prolog 1.4 is now available}}\\ % [4mm]
\end{LARGE}
\begin{Large}
{\tt ciao@dia.fi.upm.es}\\
{\tt http://www.clip.dia.fi.upm.es/Software}\\ % [2mm]
\end{Large}
\end{center}
]

% \large

\inbf{Ciao} is a \index{public domain} programming environment which
supports the development and efficient compilation of \enfat{logic
programs}, \enfat{constraint logic programs} (CLP), \enfat{functional logic
programs}, and \enfat{object-oriented logic programs}. In particular,
\inbf{Ciao} includes standard \inbf{ISO-Prolog} as a sublanguage and a
program development environment similar to that of traditional Prolog
and CLP implementations. However, \inbf{Ciao} extends the Prolog
language and improves on traditional Prolog programming environments
in a number of significant ways:

\begin{itemize} 

\item \inbf{Ciao} offers support for \enfat{programming in the large} with
      a robust module/object system, module-based automatic
      incremental compilation (with no need for makefiles), an
      assertion language for declaring (\enfat{optional}) program
      properties (including types and modes), automatic static
      inference and static/dynamic checking of such assertions, etc.

\item \inbf{Ciao} also offers support for \enfat{programming in the small}
      producing small executables (including only those builtins used
      by the program) and support for writing scripts in Prolog.

\item The \inbf{Ciao} compiler generates several forms of
      architecture-independent and stand-alone executables.  Library
      modules can be compiled into compact bytecode or C source files,
      and linked statically, dynamically, or autoloaded.

\item \inbf{Ciao} supports concurrency (threads), distributed execution,
      and parallel execution.

\item The \inbf{Ciao} programming environment includes a large number of
      \enfat{libraries} providing a wide range of additional
      functionality, from WWW programming to support for several
      control rules.

\item The \inbf{Ciao} programming environment also includes
      \apl{lpdoc}, a tool for generating documentation automatically
      for Prolog programs adorned with (\inbf{Ciao}) assertions and
      special, machine-readable comments.

\end{itemize}

The \inbf{Ciao} language has been designed from the ground up to be
highly extensible (in a modular way) and to allow modular program
development, global program analysis, and static debugging and
optimization via source to source program transformation, including
automatic parallelization.  The latter tasks are performed by the
\inbf{Ciao preprocessor} (\apl{ciaopp}, distributed separately). The
\inbf{Ciao} system also includes a rich \apl{emacs} interface and a
number of execution visualization tools.

\vfill

\inbf{Ciao} is distributed under the \concept{GNU General Public
License}, version 2, as published by the Free Software Foundation,
Inc., 675 Mass Ave, Cambridge, MA 02139, USA.

\vfill

\noindent
{\bf Contact info:}
\ \\
%% \begin{itemize}
%% \item Postal address is: \\
%%     \ \\
    \htmladdnormallink{\texttt{http://www.clip.dia.fi.upm.es}}{http://www.clip.dia.fi.upm.es}\\ 
    \htmladdnormallink{\texttt{clip@clip.dia.fi.upm.es}}{mailto:clip@clip.dia.fi.upm.es}\\ 
    Manuel Hermenegildo\\
    Facultad de Inform\'{a}tica -- UPM\\
    E-28660 Boadilla del Monte, Madrid, SPAIN

\end{document}
