% -----------------------------------------------------------
% Brochure for Ciao 1.10 announcement (for ICLP'03 and later)
% -----------------------------------------------------------
\documentclass{article}

%\usepackage{utopia}
\usepackage{twocolumn}
\usepackage{fullpage}
\usepackage{ciaocopy}
\textwidth 15.8cm
\tolerance=10000

\makeatletter
\def\subsection{\@startsection{subsection}{2}{\z@}{-1.0ex plus -1ex minus 
 -.2ex}{0.7ex plus .2ex}{\normalsize\bf}}
\makeatother

\renewcommand{\baselinestretch}{0.95}
\newcommand{\myitems}{
\renewcommand{\baselinestretch}{0.5}
\parsep 0pt \parindent 0pt \itemsep -4pt \topsep -3pt }

\input{psfig}
\input{/home/clip/Papers/clip_description/logo.tex}
\newcommand{\ciao}{\psfig{figure=ciao_s.ps,width=2.5cm}}

\begin{document}
\textheight = 670pt
\pagestyle{empty}

\twocolumn[
\noindent
\hbox{
\begin{Huge}
{\bf The 
\begin{minipage}[c]{0.3\textwidth}
\ciao
\end{minipage}
1.10 Prolog System}\\
\end{Huge}
%\engcliplogo
}
\centerline{Ciao is \emph{free software} distributed under GNU
  licenses}  
\vspace{0.5\baselineskip}
]

\noindent
Ciao provides a modern, high-performance, \emph{ISO-standard} Prolog
system, with very significant extensions: its novel modular design
allows both \emph{restricting} and \emph{extending} the language. This
makes it possible to work with \emph{fully declarative subsets} of
Prolog and also to \emph{extend} these subsets (or ISO-Prolog) both
syntactically and semantically.  Most importantly, these restrictions
and extensions can be activated separately on each program module so
that several extensions can coexist in the same application for
different modules.

Ciao currently includes extensions for feature terms (records),
functional logic programming, higher-order (with predicate
abstractions), constraints, object-oriented programming, persistence,
several control rules, concurrency, distributed execution (agents),
concurrent built-in database, and parallel execution.  Libraries also
support WWW programming, sockets, external interfaces, etc.\ (see the
\textbf{Main Features} section.)

\emph{Programming in the large} is supported thanks to a robust
module/class system, module-based automatic incremental compilation
(with no need for makefiles), an assertion language for declaring
(\emph{optional}) program properties (including types, modes,
determinacy, non-failure, cost, ...), automatic static inference and
static/dynamic checking of such assertions, program optimization and
parallelization, and powerful automatic documentation generation (the
latter tasks performed by the \emph{CiaoPP preprocessor} and the
\emph{LPdoc autodocumenter}).  It is the novel modular design of Ciao
which enables modular program development, effective global program
analysis, and static debugging and optimization via source to source
program transformation.

\emph{Programming in the small} is supported by having reduced size
executables, which only include those builtins and libraries used by
the program, and by supporting Prolog \emph{scripts}.

The compiler generates several forms of architecture-independent and
stand-alone executables. Program modules can be 
%% compiled into compact bytecode and 
linked statically, dynamically, or autoloaded. The executables
generated are very competitive in both performance and size with all
current commercial and academic Prolog systems.  
%% Optimizing compilation to 100\% native code is undergoing work.

The programming environment also offers a rich graphical interface
(with direct access to top-level/debugger, preprocessor, and
autodocumenter, based on Emacs), an embeddable source-level debugger
with breakpoints, and execution visualization tools.

\medskip
%\hrule

\subsection*{Why do we call it Ciao?}
%After reading this brochure 
The sharp reader may have already seen the logic behind the 'Ciao
Prolog' name. Ciao means both \emph{hello} and \emph{goodbye}, and
Ciao Prolog intends to be on one hand a really excellent, freely
available ISO-Prolog system, for both academic and industrial use
(including introducing users to Prolog and to classical constraint and
logic programming) --- the \emph{hello} part.  But Ciao is also a
new-generation, multiparadigm programming language and program
development system which goes well beyond Prolog and other classical
logic programming languages. And it has the advantage (when compared
to other systems) that it does so while allowing full Prolog
compatibility when needed.

%% Ciao is a \emph{next generation} logic programming environment.

\medskip\hrule

\subsection*{Main Features of Ciao:}
\begin{itemize}
\myitems
% General:
\item Efficient, high-performance, bytecode-based engine, with garbage
  collection, unbound precision integer arithmetic, built-in
  concurrency capabilities, and many other features.
%TA:
\item  Full compliance with 
%International Standard ISO/IEC 13211-1
  ISO-Prolog.
\item Generation of \textbf{multi-architecture} executables: Linux,
  {\Large $*$}NIX, Mac OS X, Win32.
\item User-friendly installation on all platforms.
\item Advanced, graphical development environment with
  integrated source-level debugging, syntax highlighting, 
  on-line help, etc. (\texttt{emacs}-based).
%% \item Source code debugger (with code highlighting).
\item Debugger embeddable in executables.
%TA:
\item Exception handling.
\item Source code autodocumenter (with a menu-based interface).
\item Flexible customization of library paths and path aliases.
%
% Module system/Libraries:
\item New generation, robust module system.
\item Modular clp(R) / clp(Q).
\item Higher-order syntax and predicate abstractions.
% \item Library of higher-order list predicates.
% \item Libraries for and-fair breadth-first and iterative deepening.
\item Extensive, built in, and modular code expansion facilities
  (macros) with operators local to modules.
\item Attributed variables, DCGs.
\item Backwards compatibility libraries (DEC-10 IO, Quintus-like
  internal database, etc.). 
\item Libraries of (commented) types, modes, and other properties to
  be used in assertions (for debugging and documentation generation).
\item Assertion-based declaration of meta-predicates.
%
% Programming styles:
\item Several execution strategies available: Andorra, breadth-first,
  iterative deepening.
\item Object-oriented extensions.
\item Bidirectional foreign interfaces: C, Java, TclTk, ProVRML.
\item Interface to SQL, relational databases.
\item Concurrency / multiengine primitives (\&Prolog-like). 
%TA:
\item Full thread support in Linux / Unix / Mac OS X / Win32.
\item Access to operating system resources.
%TA:
\item Platform independent socket communication.
%TA:
\item Platform independent Prolog Makefiles.
\item Delay predicates (when/2, freeze/1).
\item Active modules (distr./agent programming).
\item Remote loading of modules.
\item Web/Internet programming: \emph{PiLLoW} library.
\item WebDB WWW database interfacing.
\item Programmer-transparent 
  persistent predicates (files or relational database storage).
\item Execution of fuzzy logic-based programs.
%
% Engine/Compiler features:
\item Ciao CGI executables under IIS.
\item Number of clauses/predicates essentially unbound.
\item Unbound atom size.
\item Fast creation of new unique atoms.
\item Fast writing/reading (marshalling and unmarshalling) of terms.
\item Compressed object code/executables. % on demand. 
\item Fast compilation and startup. 
\item Incremental stand-alone compiler with separate compilation.
\item Automatic (re)compilation of foreign files.
\item Attributed variables.
\item Extensive, up to date documentation in multiple formats.
\end{itemize}

\hrule

\subsection*{What's new / improved in 1.10?}
\begin{itemize}
\myitems
\item Full ``Classic Prolog'' is now default behavior.
\item New Mac OS X kernels supported. Improved compilation and
  installation in all platforms. Compatibility with newer versions of
  Cygwin. 
%\item Double-click startup of programming environment.
\item Error location extended. Error marks cleared automatically. % also when generating docs.
\item Automatic / manual location of errors produced by Ciao
  tools now customizable.
\item Syntax-based coloring greatly improved.  Colors changed to make
  programs more readable. Now also working on ASCII terminals (for
  newer versions of \texttt{emacs}).
\item Faces for syntax-based highlighting more customizable.
\item Added new tool-bar button (and binding) to refontify
  block/buffer.
\item New icons and facilities in the environment for the
  preprocessor.
%\item Emacs-based environment improved.
\item Improved \texttt{emacs} inferior (interaction) mode for Ciao and CiaoPP.
\item Presentation of Ciao preprocessor output improved.
\item Reorganized menus: help and customization grouped in separate
  menus.
\item \texttt{XEmacs} compatibility improved (thanks to A. Rigo).
\item File aliasing to internal streams added.
\item The \texttt{make} library has been improved.
\item PiLLoW library improved in many ways.
\item Tcl/Tk library improved and made more robust.
\item Improved assertions package.
\item Atom-to-term conversion library improved.
\item Exceptions in active modules improved.
\item DaVinci error processing improved.
\item Persistent predicates improved.
\item File locking capabilities included.
\item New input/output facilities added to socket-related library.
\item Added treatment of\- operators and 
      \texttt{module:pred} calls to pretty-printer.
\item Updated reporting of read syntax errors.
\item Unbound length atoms now supported in all cases.
\item C interface \texttt{.h} files reachable through a more standard
  location (thanks to R. Bagnara).
\item Compatibility with command line options of newer \texttt{gcc}
  versions.
\item Solved conflicts in predicate reexportation.
\item Replication of clauses in some cases corrected (thanks to
  S. Craig).
\item Got rid of several SEGV problems.
\item The number of significant decimal digits to be printed is now
  computed accurately.
\item The reader has improved its behavior w.r.t. non-terminated
  comments.
  
\item The \texttt{xref} library treats void references now.
\item Foreign predicates are now automatically declared as
  implementation-defined ones.
\item New utilities to build module frontiers.
\item Facts defined in external files can now be automatically cached
  on-demand.
\item And, in general, new features in many libraries.
\end{itemize}

\hrule

\subsection*{Additional contributions:\\ (in development or beta state)}
\begin{itemize}
\myitems
\item clp(FD).
\item XML querying and transformation to Prolog.
\item XDR schema to HTML forms utility.
\item Bidirectional list traversal library.
\item Interface to GnuPlot.
\item Libraries for execution time profiling.
\end{itemize}
%% \noindent
%% \textbf{Ciao Prolog 1.10 will be available in short} 

%% \noindent
%% Release candidate versions are being posted at the Ciao Prolog WWW
%% site.  We very much appreciate feedback on Ciao Prolog!

%% \medskip
%% \hrule

\hrule

\subsection*{Contact / download info:}
\texttt{http://cliplab.org}\\ 
\texttt{http://ciaoprolog.org}\\ 
\texttt{ciao@clip.dia.fi.upm.es}\\ [2mm]
The CLIP Group\\
Tecnical University of Madrid, Spain\\
University of New Mexico, USA\\
%% Facultad de Inform\'{a}tica -- UPM\\
%% E-28660 Boadilla del Monte, Madrid, SPAIN

\hrule

\end{document}
