% -----------------------------------------------------------
% Brochure for Ciao 1.8 announcement (for ICLP'01)
% -----------------------------------------------------------
\documentclass{article}

\usepackage{a4wide}
\usepackage{twocolumn}
\topmargin-1cm
\tolerance=10000

\makeatletter
\def\subsection{\@startsection{subsection}{2}{\z@}{-1.0ex plus -1ex minus 
 -.2ex}{0.7ex plus .2ex}{\normalsize\bf}}
\makeatother

\renewcommand{\baselinestretch}{0.95}
\newcommand{\myitems}{
\renewcommand{\baselinestretch}{0.5}
\parsep 0pt \parindent 0pt \itemsep -4pt \topsep -3pt }

\input{psfig}
\input{/home/clip/Papers/clip_description/logo.tex}
\newcommand{\ciao}{\psfig{figure=ciao_s.ps,width=2.5cm}}

\begin{document}

\pagestyle{empty}

\twocolumn[
\noindent
\hbox{
\begin{Huge}
{\bf The 
\begin{minipage}[c]{0.3\textwidth}
\ciao
\end{minipage}
1.8 Prolog System}\\
\end{Huge}
%\engcliplogo
}
\centerline{Ciao is \emph{free software} distributed under GNU
  licenses}  
\vspace{0.5\baselineskip}
]

Ciao is a complete Prolog system, supporting \emph{ISO-Prolog}, but
its novel modular design allows both \emph{restricting} and
\emph{extending} the language. This makes it possible to work with
\emph{fully declarative subsets} of Prolog and also to \emph{extend}
these subsets (or ISO-Prolog) both syntactically and semantically.
Most importantly, these restrictions and extensions can be activated
separately on each program module so that several extensions can 
coexist in the same application for different modules.

Ciao currently includes extensions for feature terms (records),
functional logic programming, higher-order (with predicate
abstractions), constraints, object-oriented programming, persistence,
several control rules (breadth-first search, iterative deepening,
...), concurrency (threads/engines), distributed execution (agents),
concurrent built-in database, and parallel execution.  Libraries also
support WWW programming, sockets, external interfaces (C, Java, TclTk,
relational databases, etc.), etc.

\emph{Programming in the large} is supported thanks to the robust
module/class system, module-based automatic incremental compilation
(with no need for makefiles), an assertion language for declaring
(\emph{optional}) program properties (including types, modes,
determinacy, non-failure, cost, ...), automatic static inference and
static/dynamic checking of such assertions, program optimization and
parallelization, and powerful automatic documentation generation (the
latter tasks performed by the \emph{CiaoPP preprocessor} and the
\emph{LPdoc autodocumenter}).  It is the novel modular design of Ciao
which enables modular program development, effective global program
analysis, and static debugging and optimization via source to source
program transformation.

\emph{Programming in the small} is supported by having reduced size
executables, which only include those builtins and libraries used by
the program, and by supporting
% \emph{scripts} written in Prolog.
Prolog \emph{scripts}.

The compiler generates several forms of architecture-independent and
stand-alone executables. Program modules can be compiled into compact
bytecode and linked statically, dynamically, or autoloaded. The
executables generated are very competitive in both performance and
size with all current commercial and academic Prolog systems.

The programming environment also offers a rich \texttt{emacs}
interface (with direct access to top-level/debugger, preprocessor,
and autodocumenter), embeddable source-level debugger with
breakpoints, and some execution visualization tools. 

\subsection*{Why Ciao?}
%After reading this brochure 
The sharp reader may have already seen
the logic behind the 'Ciao Prolog' name. Ciao is an interesting
word which means both \_hello\_ and \_goodbye\_. Ciao Prolog intends to
be a really good, all-round, freely available ISO-Prolog system
which can be used as a classical Prolog, in both academic and
industrial environments (and, in particular, to introduce users to
Prolog and to constraint and logic programming) --the \_hello\_ part.
But Ciao is also a new-generation, multiparadigm programming
language and program development system which goes well beyond
Prolog and other classical logic programming languages. And it has
the advantage (when compared to other systems) that it does so
while keeping full Prolog compatibility when needed.

Ciao is a \emph{next generation} logic programming environment.

\subsection*{What's new in 1.8?}
\begin{itemize}
\myitems
\item Support for Mac OS X, based on the Darwin kernel.
\item Initial compilation support for Linux on Power PC
        (contributed by Paulo Moura).
\item Workaround for incorrect C compilation while using newer
         ($>$ 2.95) gcc compilers.
\item Stablished a (sensible) execution ordering of initialization
  directives.
\item Fixed bugs in the toplevel: behaviour of {\tt module:main} calls
  and module initialization.
\item Higher-order syntax is now a package.
\item Improved model for predicate abstractions.
\item Fuzzy Prolog package.
\item Remote loading of modules.
\item Andorra style execution package.
\item Iterative deepening execution package.
\item Hooks for pruning on deterministic computations.
\item Java interface implementation greatly improved.
\item Interface to Tcl/Tk programs.
\item Object-based interface to Tcl/Tk graphical objects.
\item ProVRML: reads VRML code and translates it into Prolog terms,
  and the other way around.
\item SQL interface now  works with more databases (e.g., mysql).
\item More predicates in the interface to the underlying system /
  operating system. 
\item WebDB: utilities to create WWW-based database interfaces.
\item Support of environment in emacs 21.1.
\item Specific tool bar now available, with icons for main fuctions 
  (works from emacs 21.1 on).
\item Inferior modes for Ciao and CiaoPP improved.
\item New interactive entry points ({\sl M-x}): ciao, prolog, ciaopp.
\item Better tracking of last inferior buffer used.
\item Context menu in Windows can now load a file into the toplevel.
\item Updated Windows installation in order to run CGI executables.
\item Added new directories found in recent Linux distributions to
  INFOPATH. 
\item No need of a layout char to end Prolog files.
\item Syntax errors now show every time the code is used without
  change. 
\item Redirection of stdin/stdout/stderr from within Ciao Prolog
  programs. 
\item Library for cross-references among Prolog files.
\item New predicates to create new concurrent predicates on-the-fly.
\item Bugs corrected in sockets library, fast read/write, and exec/$*$.
\item New active module protocol to publish addresses through the Web.
\item Expansion and meta predicates improved.
\item Improved foreign interface.
\end{itemize}

\vfill

\subsection*{Contact / download info:}

\texttt{http://cliplab.org}\\ 
\texttt{http://ciaoprolog.org}\\ 
\texttt{ciao@clip.dia.fi.upm.es}\\ 
The CLIP Group\\
Facultad de Inform\'{a}tica -- UPM\\
E-28660 Boadilla del Monte, Madrid, SPAIN

\newpage

\subsection*{Main Features of Ciao:}
\begin{itemize}
\myitems
% General:
\item Multi-architecture executables: Linux,
  {\Large $*$}NIX, Mac OS X, Win32.
\item User-friendly installation procedure on all platforms.
\item Advanced, emacs-based, program development environment with
  source-level debugging.
\item Debugger embeddable in executables.
\item Source code autodocumenter (with a menu-based interface).
\item Flexible customization of library paths and path aliases.
%
% Module system/Libraries:
\item New generation, robust module system.
\item Modular clp(R) / clp(Q).
\item Higher-order syntax for predicate abstractions.
% \item Library of higher-order list predicates.
% \item Libraries for and-fair breadth-first and iterative deepening.
\item Extensive, built in, and modular code expansion facilities
  (macros). 
\item Attributed variables, DCGs.
\item Backwards compatibility libraries (DEC-10 IO, Quintus-like
  internal database, etc.). 
\item Libraries of (commented) types, modes, and other properties to
  be used in assertions (for debugging and documentation generation).
\item Assertion-based declaration of meta-predicates.
%
% Programming styles:
\item Object oriented extension.
\item Bidirectional foreign interfaces: C, Java, TclTk.
\item Interface to SQL, relational databases.
\item (\&Prolog-like) concurrency / multiengine primitives. 
\item Full thread support in Linux / Unix / Mac OS X / Win32.
\item Delay predicates (when/2, freeze/1).
\item Active modules (distributed/agent programming).
\item Web/Internet programming: \emph{Pillow} library.
\item Persistent logic database (persistent predicates).
\item Fuzzy logic.
%
% Engine/Compiler features:
\item Ciao CGI executables under IIS.
\item Unbound atom size.
\item Fast creation of new unique atoms.
\item \#~of clauses/pred.~essentially unbound.
\item Fast writing/reading (marshalling and unmarshalling) of terms.
\item Compressed object code/executables. % on demand. 
\item Fast compilation and startup.
\item Incremental stand-alone compiler.
\item Automatic (re)compilation of foreign files.
\item Extensive, up to date documentation.
\end{itemize}

\end{document}
