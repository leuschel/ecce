% -----------------------------------------------------------
% Brochure for Ciao 1.6 announcement (sent to ALP newsletter)
% -----------------------------------------------------------
\documentclass{article}

\usepackage{a4wide}
\usepackage{twocolumn}
\topmargin-1cm
\tolerance=10000

\makeatletter
\def\subsection{\@startsection{subsection}{2}{\z@}{-1.0ex plus -1ex minus 
 -.2ex}{0.7ex plus .2ex}{\normalsize\bf}}
\makeatother

\renewcommand{\baselinestretch}{0.95}
\newcommand{\myitems}{
\renewcommand{\baselinestretch}{0.5}
\parsep 0pt \parindent 0pt \itemsep -4pt \topsep -3pt }

\input{psfig}
\input{/home/clip/Papers/clip_description/logo.tex}
\newcommand{\ciao}{\psfig{figure=ciao_s.ps}}

\begin{document}

\pagestyle{empty}

\twocolumn[
\hbox{
\ciao~~~~~~~~
\begin{minipage}[b]{0.5\textwidth}
\begin{LARGE}
{\bf Announcing Version 1.6 of the Ciao System}\\[-0.25\baselineskip]
\end{LARGE}
\end{minipage}
%\engcliplogo
}
\vspace{0.5\baselineskip}
]

Ciao is a \emph{public domain}, \emph{next generation} logic
programming environment.

Ciao is a complete Prolog system, supporting \emph{ISO-Prolog}, but
its novel modular design allows both \emph{restricting} and
\emph{extending} the language. This makes it possible to work with
\emph{fully declarative subsets} of Prolog and also to \emph{extend}
these subsets (or ISO-Prolog) both syntactically and semantically.
Most importantly, these restrictions and extensions can be activated
separately on each program module so that several extensions can 
coexist in the same application for different modules.

Ciao currently includes extensions for feature terms (records),
functional logic programming, higher-order (with predicate
abstractions), constraints, object-oriented programming, persistence,
several control rules (breadth-first search, iterative deepening,
...), concurrency (threads/engines), distributed execution (agents),
concurrent built-in database, and parallel execution.  Libraries also
support WWW programming, sockets, external interfaces (C, Java, TclTk,
relational databases, etc.), etc.

\emph{Programming in the large} is supported thanks to the robust
module/class system, module-based automatic incremental compilation
(with no need for makefiles), an assertion language for declaring
(\emph{optional}) program properties (including types, modes,
determinacy, non-failure, cost, ...), automatic static inference and
static/dynamic checking of such assertions, program optimization and
parallelization, and powerful automatic documentation generation (the
latter tasks performed by the \emph{CiaoPP preprocessor} and the
\emph{LPdoc autodocumenter}).  It is the novel modular design of Ciao
which enables modular program development, effective global program
analysis, and static debugging and optimization via source to source
program transformation.

\emph{Programming in the small} is supported by having reduced size
executables, which only include those builtins and libraries used by
the program, and by supporting
% \emph{scripts} written in Prolog.
Prolog \emph{scripts}.

The compiler generates several forms of architecture-independent and
stand-alone executables. Program modules can be compiled into compact
bytecode and linked statically, dynamically, or autoloaded. The
executables generated are very competitive in both performance and
size with all current commercial and academic Prolog systems.

The programming environment also offers a rich \texttt{emacs}
interface (with direct access to top-level/debugger, preprocessor,
and autodocumenter), embeddable source-level debugger with
breakpoints, and some execution visualization tools. 

\subsection*{1.4$\rightarrow$1.6 improvements:}

\begin{itemize}
\myitems
\item Source-level debugger in emacs, breakpts.
% \item Emacs environment improved, added menus for Ciaopp and LPDoc.
\item Debugger embeddable in executables.
% \item Stand-alone executables available for UNIX-like o.s. 
\item Many improvements to emacs interface.
\item Menu-based interface to autodocumenter.
\item Threads now available in Win32.
\item Many improvements to threads.
\item Modular clp(R) / clp(Q).
% \item Libraries for and-fair breadth-first and iterative deepening.
\item Improved syntax for pred.~abstractions.
\item Library of higher-order list predicates.
\item Better code expansion facilities (macros).
\item New delay predicates (when/2).
\item Compressed object code/executables. % on demand. 
\item The size of atoms is now unbound.
\item Fast creation of new unique atoms.
\item \#~of clauses/pred.~essentially unbound.
\item Delayed goals with freeze restored.
\item Faster compilation and startup.
\item Much faster fast write/read. 
\item Improved documentation.
\item Other new libraries. 
\item Improved installation/deinstallation.
\item Many improvements to autodocumenter.
\item Many bug fixes.
\end{itemize}

\subsection*{1.0$\rightarrow$1.4 improvements:}

\begin{itemize}
\myitems
\item Automatic (re)compilation of foreign files.
\item New version of O'Ciao objects, 
      with improved performance. 
\item New  (\&Prolog-like) concurrency / multiengine primitives. 
\item Support for {\em predicate abstractions}.
\item Reexportation of predicates. 
\item Precedence of importations changed: last one is now higher. 
\item Modules can implicitly export all preds.
\item Improved documentation.
\end{itemize}

\subsection*{Contact / download info:}

\texttt{http://www.clip.dia.fi.upm.es}\\ 
\texttt{http://www.clip.dia.fi.upm.es/Software}\\ 
\texttt{clip@clip.dia.fi.upm.es}\\ 
Manuel Hermenegildo / The CLIP Group\\
Facultad de Inform\'{a}tica -- UPM\\
E-28660 Boadilla del Monte, Madrid, SPAIN

\end{document}
